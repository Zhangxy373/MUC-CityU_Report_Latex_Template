%!TeX program = xelatex
\documentclass[12pt,hyperref,a4paper,UTF8]{ctexart}
\usepackage{MUCReport}

%%-----------------正文开始------------------%%
\begin{document}
%%-------------------封面--------------------%%
\cover
%%------------------摘要---------------------%%
%\begin{abstract}
%
%在此填写摘要内容
%
%\end{abstract}

\thispagestyle{empty} % 首页不显示页码

%%--------------------------目录页------------------------%%
\newpage
\tableofcontents
\thispagestyle{empty} % 目录不显示页码

%%------------------------正文页从这里开始-------------------%
\newpage
\setcounter{page}{1} % 让页码从正文开始编号

%%可选择这里也放一个标题
%\begin{center}
%    \title{ \Huge \textbf{{标题}}}
%\end{center}

\section{模板说明}
本模板主要适用于一些课程的平时论文以及期末论文,默认页边距为2.5cm,中文宋体,英文Times New Roman,字号为12pt(小四)。

编译方式:\verb|xelatex -> bibtex -> xelatex*2|


默认模板文件由以下四部分组成:
\begin{itemize}
    \item \texttt{main.tex} 主文件
    \item \texttt{reference.bib} 参考文献,使用bibtex
    \item \texttt{MUCReport.sty} 文档格式控制,包括一些基础的设置,如页眉、标题、学院、学号、姓名等
    \item \texttt{figures} 放置图片的文件夹
\end{itemize}

第一次使用时需前往\texttt{MUCReportReport.sty} 对标题、姓名、学号、页眉等进行设置,设置完后即可一劳永逸,封面LOGO亦可替换。

默认带有封面页以及目录页,页码从目录页开始。

\section{一些插入功能}
\subsection{插入公式}
行内公式$v-\varepsilon+\phi=2$。

插入行间公式如\autoref{Euler}:
\begin{equation}
    v-\varepsilon+\phi=2
    \label{Euler}
\end{equation}

\subsection{插入图片}
MUC校徽如\autoref{MUC}所示,注意这里使用了\verb|~\autoref{}|命令,也就是会自动生成“图”“式”等前缀,无需手动输入。

\begin{figure}[!htbp]
    \centering
    \includegraphics[width =.5\textwidth]{figures/muc.png}
    \caption{中央民族大学}
    \label{MUC}
\end{figure}

\begin{figure}[htbp]
\centering
\subcaptionbox{澳门城市大学计算机95\label{fig:11}}%
  {\includegraphics[width=0.48\textwidth]{figures/CityU.png}}\hfill
\subcaptionbox{中央民族大学软科103\label{fig:22}}%
  {\includegraphics[width=0.48\textwidth]{figures/muc.png}}
\caption{很好的两所学校:)}
\label{fig:ds1}
\end{figure}

这是两张图片联排的样式。

\begin{wrapfigure}{r}{0.5\textwidth}%靠文字内容的右侧
\centering
\includegraphics[width=0.45\textwidth]{figures/CityU.png}
\caption{这个是文字环绕式排版} \label{fig1}
\end{wrapfigure}
文字环绕式的图片一定要放在想插入的段落的前面,否则Latex就会自作主张发疯。

插入上面图片的代码:

\begin{verbatim}
    \begin{figure}[!htbp]
        \centering
        \includegraphics[width =.5\textwidth]{}
        \caption{中央民族大专}
        \label{RUC}
    \end{figure}
\end{verbatim}

\subsection{插入文本框}
本模板定义了一个圆角灰底的文本框,使用简化命令\verb|\tbox{}|即可,如果你不喜欢,可以前往 \texttt{MUCReport.sty}对其进行修改。

\tbox{
    这是一个圆角灰底的文本框
}

\subsection{插入表格}
本模板文件如表~\ref{doc} 所示。
\begin{table}[!htbp]
    \caption{本模板文件组成}
    \centering
    \begin{tabular}{l  | l}
    \hline
        文件名 & 说明 \\
        \hline
        \texttt{main.tex}  & 主文件 \\
        \texttt{reference.bib} & 参考文献 \\
        \texttt{MUCReport.sty}  & 文档格式控制\\
        \texttt{figures}  & 图片文件夹 \\
        \hline
    \end{tabular}
    \label{doc}
\end{table}


\begin{table*}[!ht]
\caption{三线表}\label{tab:三线表}
\centering
\begin{tabularx}{\textwidth}{@{}lXXXX@{}} % First column left-aligned, others flexible
\toprule
Model & Recall & Precision & F1 score & Accuracy \\
\midrule
321 & 0.1 & 0.1 & 0.1 & 0.1 \\
123 & 0.1 & 0.1 & 0.1 & 0.1 \\
321 & 0.1 & 0.1 & 0.1 & 0.1 \\
123 & 0.1 & 0.1 & 0.1 & 0.1 \\
\bottomrule
\end{tabularx}
\end{table*}
%\section{定理环境}
%\begin{Theorem}
%\end{Theorem}
%
%\begin{Lemma}
%\end{Lemma}
%
%\begin{Corollary}
%\end{Corollary}
%
%\begin{Proposition}
%\end{Proposition}
%
%\begin{Definition}
%\end{Definition}
%
%\begin{Example}
%\end{Example}
%
%\begin{proof}
%\end{proof}

\subsection{插入高亮代码块}
\textbf{MATLAB source code:}
\lstinputlisting[language=Matlab]{./code/matlab.m}
\textbf{Python source code:}
\lstinputlisting[language=Python]{./code/python.py}
%\textbf{Pseudocode:} %伪代码
%\lstinputlisting[language=C]{./code/Pseudocode.cpp}

利用\verb|lstlisting| 配置
\begin{lstlisting}[style=CPP, title="c++代码"]
#include <iostream>
#include <array>
int main()
{
    constexpr int MAX = 100;
    std::array<int, MAX> arr;
}  
\end{lstlisting}

\begin{lstlisting}[style=Java, title="Java代码"]
public void addAdvertisement(String company, String ad_Category, String ad_Type, String ad_Price)
{
    int price = Integer.parseInt(ad_Price);
    ad = new Advertisement(company, ad_Category, ad_Type, price);
    adList.add(index, ad);
    index++;
    anDM = getDefaultDirectoryManager();
    ActorTuple tuple = new ActorTuple(getActorName(), "advertiser",
    company, ad_Category, ad_Type, price, index-1);
    send(anDM, "register", tuple);
}
\end{lstlisting}

\begin{lstlisting}[style=Python, title="Python代码"]                
import random
import collections
Card = collections.namedtuple('Card', ['rank', 'suit'])

class FrenchDesk:
    ranks = [str(n) for n in range(2, 11)] + list('JQKA')
    suits = 'spades diamonds clubs hearts'.split()
    
    def __init__(self):
        self._cards = [Card(rank, suit) for rank in self.ranks for suit in self.suits]
        
    def __len__(self):
        return len(self._cards)
        
    def __getitem__(self, position):
        return self._cards[position]
desk = FrenchDesk()
\end{lstlisting}

\subsection{插入参考文献}
直接使用\verb|\cite{}|即可。

例如:

   \textit{ 此处引用了文献\cite{0Isaac}。此处引用了文献\cite{2016The}}

引用过的文献会自动出现在参考文献中。

\section{写在最后}
\subsection{发布地址}
\begin{itemize}
    \item Github: \url{https://github.com/Zhangxy373/MUC-CityU_Report_Latex_Template}
    \item Overleaf:  \url{}
\end{itemize}

%%----------- 参考文献 -------------------%%
%在reference.bib文件中填写参考文献,此处自动生成

\reference


\end{document}